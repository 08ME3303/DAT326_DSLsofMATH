% DSLsofMath-PD.tex
\begin{hcarentry}[updated]{DSLsofMath}
\report{Patrik Jansson}%11/17
\status{active development}
\participants{Cezar Ionescu, Daniel Heurlin}
\makeheader

``Domain Specific Languages of Mathematics'' is a project at
\href{http://www.chalmers.se/en/Pages/default.aspx}{Chalmers} and
\href{http://www.gu.se/english}{UGOT} which lead to a new BSc level course of
the same name, including accompanying material for learning and applying
classical mathematics (mainly basics of real and complex analysis).
%
The main idea is to encourage the students to approach mathematical domains
from a functional programming perspective:
%
to identify the main functions and types involved and, when necessary, to
introduce new abstractions;
%
to give calculational proofs;
%
to pay attention to the syntax of the mathematical expressions;
%
and, finally, to organize the resulting functions and types in domain-specific
languages.

The second instance of the course was carried out Jan-March 2017 at Chalmers
and the course material is available on
\href{https://github.com/DSLsofMath/DSLsofMath}{github}.
%
The next step (ongoing work) is to write up the lecture notes as a ``book''
during the autumn, in preparation for the next instance of the course early
2018.
%
Contributions and ideas are welcome!

\FurtherReading
\begin{compactitem}
\item \href{https://github.com/DSLsofMath}{DSLsofMath (github organisation)}
\item \href{https://github.com/DSLsofMath/tfpie2015}{TFPIE 2015 paper}
\item \href{https://github.com/DSLsofMath/DSLsofMath/tree/master/L/snapshots}{Latest snapshot of the DSLsofMath Lecture Notes (work in progress)}
\item \href{https://github.com/DSLsofMath/DSLsofMath/blob/master/Exam/2017-03/}{Exam 2017 with solutions}
\end{compactitem}
\end{hcarentry}
